\documentclass[12pt,a4paper]{defesa}
\begin{document}
% \bsi \tads \mestrado ou \doutorado
\mestrado
% Nome do autor ou da autora (nesse caso use \autora{nome}
\autor{Ze Mane}
% Universidade
\instituicao{Universidade Estadual de Campinas}
% Faculdade. Comente se não houver uma faculdade específica
\faculdade{Faculdade de Tecnologia}
% Data da apresentação
\data{20/Jan/2019, 16h}
% Título do Trabalho
\titulo{Título teste}
%
% Mantenha o comando a seguir para imprimir as informações
\paginasiniciais
%
% Início do texto
\section*{Considerações Gerais}
A seguir estão algumas considerações gerais a respeito do seu trabalho. É importante que essas considerações sejam observadas ao efetuar as correções sugeridas. Outras observações podem estar indicadas no próprio texto.

\begin{enumerate}
    \item Primeira
    \item segunda
\end{enumerate}

\section*{Dúvidas e sugestões}
Algumas dúvidas e sugestões sobre o texto estão resumidas na tabela a seguir.

\begin{longtable}{@{}ccm{.56\paperwidth}@{}}%ccp{11cm}}
\toprule\toprule
\textbf{Página} & \textbf{Parágrafo} &  \textbf{Dúvida/Sugestão}\\
\midrule[\heavyrulewidth]
\endhead % all the lines above this will be repeated on every page
\bottomrule\bottomrule
\endfoot
Resumo & Primeiro & teste teste teste teste teste teste teste teste teste teste teste teste teste teste teste teste teste teste\\\hdashline
12 & \paragrafo{2} & teste teste teste teste teste teste teste teste teste teste teste teste teste teste teste teste teste teste\\
\end{longtable}


\section*{Considerações Finais}
A seguir estão as considerações finais a respeito do seu trabalho:
\begin{enumerate}
    \item Primeira
    \item segunda
\end{enumerate}
\end{document}